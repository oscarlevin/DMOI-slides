\documentclass[11pt, compress]{beamer}
\usepackage{amsmath}
\usetheme{Boadilla}
\usefonttheme[onlymath]{serif}
%get rid of navigation:
\setbeamertemplate{navigation symbols}{}


 %%%% Start PreTeXt generated preamble: %%%%% 

%% Some aspects of the preamble are conditional,
%% the LaTeX engine is one such determinant
\usepackage{ifthen}
\newcommand{\tabularfont}{}
\usepackage[xparse, raster]{tcolorbox}
\tcbset{colback=white, colframe=white}
\NewTColorBox{image}{mmm}{boxrule=0.25pt, colframe=gray, left skip=#1\linewidth,width=#2\linewidth}
\RenewTColorBox{definition}{m}{colback=teal!30!white, colbacktitle=teal!30!white, coltitle=black, colframe=gray, boxrule=0.5pt, sharp corners=downhill, titlerule = 0.25pt, title={#1}}
\RenewTColorBox{theorem}{m}{colback=pink!30!white, colbacktitle=pink!30!white, coltitle=black, colframe=gray, boxrule=0.5pt, sharp corners=downhill, titlerule = 0.25pt, title={#1}}
\RenewTColorBox{proof}{}{boxrule=0.25pt, colframe=gray, colback=white, before upper={Proof:}, after upper={\qed}}
\newcommand{\lt}{<}
\newcommand{\gt}{>}
\newcommand{\amp}{&}

%% Begin: Semantic Macros
%% To preserve meaning in a LaTeX file
%%
%% \mono macro for content of "c", "cd", "tag", etc elements
%% Also used automatically in other constructions
%% Simply an alias for \texttt
%% Always defined, even if there is no need, or if a specific tt font is not loaded
\newcommand{\mono}[1]{\texttt{#1}}
%%
%% Following semantic macros are only defined here if their
%% use is required only in this specific document
%%
%% Used for inline definitions of terms
\newcommand{\terminology}[1]{\textbf{#1}}
%% End: Semantic Macros

\renewcommand{\d}{\displaystyle}
\newcommand{\N}{\mathbb N}
\newcommand{\B}{\mathbf B}
\newcommand{\Z}{\mathbb Z}
\newcommand{\Q}{\mathbb Q}
\newcommand{\R}{\mathbb R}
\def\C{\mathbb C}
\def\U{\mathcal U}
\newcommand{\pow}{\mathcal P}
\newcommand{\inv}{^{-1}}
\newcommand{\st}{:}
\renewcommand{\iff}{\leftrightarrow}
\newcommand{\Iff}{\Leftrightarrow}
\newcommand{\imp}{\rightarrow}
\newcommand{\Imp}{\Rightarrow}
\newcommand{\isom}{\cong}

\renewcommand{\bar}{\overline}
\newcommand{\card}[1]{\left| #1 \right|}
\newcommand{\twoline}[2]{\begin{pmatrix}#1 \\ #2 \end{pmatrix}}

\newcommand{\vtx}[2]{node[fill,circle,inner sep=0pt, minimum size=4pt,label=#1:#2]{}}
\newcommand{\va}[1]{\vtx{above}{#1}}
\newcommand{\vb}[1]{\vtx{below}{#1}}
\newcommand{\vr}[1]{\vtx{right}{#1}}
\newcommand{\vl}[1]{\vtx{left}{#1}}
\renewcommand{\v}{\vtx{above}{}}

%% Graphics Preamble Entries
\usepackage{tikz, pgfplots}

\usetikzlibrary{positioning,matrix,arrows}

\usetikzlibrary{shapes,decorations,shadows,fadings,patterns}
\usetikzlibrary{decorations.markings}

\usepackage{skak} %for chessboards etc.

\def\circleA{(-.5,0) circle (1)}
\def\circleAlabel{(-1.5,.6) node[above]{$A$}}
\def\circleB{(.5,0) circle (1)}
\def\circleBlabel{(1.5,.6) node[above]{$B$}}
\def\circleC{(0,-1) circle (1)}
\def\circleClabel{(.5,-2) node[right]{$C$}}
\def\twosetbox{(-2,-1.4) rectangle (2,1.4)}
\def\threesetbox{(-2.5,-2.4) rectangle (2.5,1.4)}
\newcommand{\hexbox}[3]{
  \def\x{-cos{30}*\r*#1+cos{30}*#2*\r*2}
  \def\y{-\r*#1-sin{30}*\r*#1}
  \draw (\x,\y) +(90:\r) -- +(30:\r) -- +(-30:\r) -- +(-90:\r) -- +(-150:\r) -- +(150:\r) -- cycle;
  \draw (\x,\y) node{#3};
}

\tikzset{->-/.style={decoration={
  markings,
  mark=at position .5 with {\arrow{>}}},postaction={decorate}}}

  \newcommand{\onedot}{
    +(.5,.5) \v
  }
  \newcommand{\twodots}{
    +(.25,.25) \v +(.75,.75) \v
  }
  \newcommand{\threedots}{
  +(.25,.25) \v +(.5, .5) \v +(.75,.75) \v
  }
  \newcommand{\fourdots}{
    +(.25,.25) \v +(.25,.75) \v +(.75,.25) \v +(.75,.75) \v
  }
  \newcommand{\fivedots}{
    +(.5,.5) \v +(.25,.25) \v +(.25,.75) \v +(.75,.25) \v +(.75,.75) \v
  }
  \newcommand{\sixdots}{
    +(.25,.5) \v +(.75,.5) \v +(.25,.25) \v +(.25,.75) \v +(.75,.25) \v +(.75,.75) \v
  }
  \newcommand{\dominoborder}{
    \draw[thick, rounded corners] (0,0) rectangle (1,2);
    \draw[thin] (0,1) -- (1,1);
  }


%%%% End of PreTeXt generated preamble %%%%% 

\title{Solving Recurrence Relations}
\subtitle{(Section 2.4)}
\author{}
\date[]{}

\begin{document}
\begin{frame}
\maketitle 
\end{frame}
 
\begin{frame}
\frametitle{Overview}
\tableofcontents 
\end{frame}
 

\section{Telescoping and Iteration}
\begin{frame}
\frametitle{Telescoping}
 \terminology{Telescoping} refers to the phenomenon when many terms in a large sum cancel out - so the sum ``telescopes.'' For example:%
\begin{equation*}
(2 - 1) + (3 - 2) + (4 - 3) + \cdots + (100 - 99) + (101 - 100) = -1 + 101
\end{equation*}
because every third term looks like: \(2 + -2 = 0\), and then \(3 + -3 = 0\) and so on.
\end{frame}
 
\begin{frame}
\frametitle{}
\begin{example}[2.4.3]Solve the recurrence relation \(a_n = a_{n-1} + n\) with initial term \(a_0 = 4\).
\end{example}
\end{frame}
 
\begin{frame}
\frametitle{Iteration}
 What do you do when the left-hand side doesn't telescope?  \terminology{Iterate}: repeatedly find the next term and simplify.
 \begin{example}[2.4.4]Use iteration to solve the recurrence relation \(a_n = a_{n-1} + n\) with \(a_0 = 4\).
\end{example}
\end{frame}
 
\begin{frame}
\frametitle{}
\begin{example}[2.4.5]Solve the recurrence relation \(a_n = 3a_{n-1} + 2\) subject to \(a_0 = 1\).
\end{example}
\end{frame}
 


\section{The Characteristic Root Technique}
\begin{frame}
\frametitle{}
Suppose we want to solve  \(a_n = a_{n-1} + 6a_{n-2}\). Iteration is too complicated, but think just for a second what would happen if we \emph{did} iterate. Perhaps the solution will take the form \(r^n\) for some constant \(r\).
 
\pause \vfill 

What happens if we plug in \(r^n\) into the recursion above? We get%
\begin{equation*}
r^n - r^{n-1} - 6r^{n-2} = 0\text{.}
\end{equation*}

 
\pause \vfill 

Now solve for \(r\):%
\begin{equation*}
r^{n-2}(r^2 - r - 6) = 0\text{,}
\end{equation*}
so by factoring, \(r = -2\) or \(r = 3\) This tells us that \(a_n = (-2)^n\) is a solution to the recurrence relation, as is \(a_n = 3^n\).
 
\pause \vfill 

In fact, for any \(a\) and \(b\), \(a_n = a(-2)^n + b 3^n\) is a solution. To find the values of \(a\) and \(b\), use the initial conditions.
\end{frame}
 
\begin{frame}
\frametitle{Characteristic Roots}
 Given a recurrence relation \(a_n + \alpha a_{n-1} + \beta a_{n-2} = 0\), the \terminology{characteristic polynomial} is%
\begin{equation*}
x^2 + \alpha x + \beta
\end{equation*}
giving the \terminology{characteristic equation}:%
\begin{equation*}
x^2 + \alpha x + \beta = 0\text{.}
\end{equation*}

 
\pause \vfill 

If \(r_1\) and \(r_2\) are two distinct roots of the characteristic polynomial (i.e, solutions to the characteristic equation), then the solution to the recurrence relation is%
\begin{equation*}
a_n = ar_1^n + br_2^n\text{,}
\end{equation*}
where \(a\) and \(b\) are constants determined by the initial conditions.
\end{frame}
 
\begin{frame}
\frametitle{}
\begin{example}[2.4.6]Solve the recurrence relation \(a_n = 7a_{n-1} - 10 a_{n-2}\) with \(a_0 = 2\) and \(a_1 = 3\).
\end{example}
\end{frame}
 
\begin{frame}
\frametitle{Characteristic Root Technique for Repeated Roots}
 Suppose the recurrence relation \(a_n = \alpha a_{n-1} + \beta a_{n-2}\) has a characteristic polynomial with only one root \(r\). Then the solution to the recurrence relation is%
\begin{equation*}
a_n = ar^n + bnr^n
\end{equation*}
where \(a\) and \(b\) are constants determined by the initial conditions.
\end{frame}
 
\begin{frame}
\frametitle{}
\begin{example}[2.4.7]Solve the recurrence relation \(a_n = 6a_{n-1} - 9a_{n-2}\) with initial conditions \(a_0 = 1\) and \(a_1 = 4\).
\end{example}
\end{frame}
 

\end{document}
