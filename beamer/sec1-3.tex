\documentclass[11pt, compress]{beamer}
\usepackage{amsmath}
\usetheme{Boadilla}
\usefonttheme[onlymath]{serif}
%get rid of navigation:
\setbeamertemplate{navigation symbols}{}


 %%%% Start PreTeXt generated preamble: %%%%% 

%% Some aspects of the preamble are conditional,
%% the LaTeX engine is one such determinant
\usepackage{ifthen}
\newcommand{\tabularfont}{}
\usepackage[xparse, raster]{tcolorbox}
\tcbset{colback=white, colframe=white}
\NewTColorBox{image}{mmm}{boxrule=0.25pt, colframe=gray, left skip=#1\linewidth,width=#2\linewidth}
\RenewTColorBox{definition}{m}{colback=teal!30!white, colbacktitle=teal!30!white, coltitle=black, colframe=gray, boxrule=0.5pt, sharp corners=downhill, titlerule = 0.25pt, title={#1}}
\RenewTColorBox{theorem}{m}{colback=pink!30!white, colbacktitle=pink!30!white, coltitle=black, colframe=gray, boxrule=0.5pt, sharp corners=downhill, titlerule = 0.25pt, title={#1}}
\RenewTColorBox{proof}{}{boxrule=0.25pt, colframe=gray, colback=white, before upper={Proof:}, after upper={\qed}}
\newcommand{\lt}{<}
\newcommand{\gt}{>}
\newcommand{\amp}{&}

%% Begin: Semantic Macros
%% To preserve meaning in a LaTeX file
%%
%% \mono macro for content of "c", "cd", "tag", etc elements
%% Also used automatically in other constructions
%% Simply an alias for \texttt
%% Always defined, even if there is no need, or if a specific tt font is not loaded
\newcommand{\mono}[1]{\texttt{#1}}
%%
%% Following semantic macros are only defined here if their
%% use is required only in this specific document
%%
%% Used for inline definitions of terms
\newcommand{\terminology}[1]{\textbf{#1}}
%% End: Semantic Macros

\renewcommand{\d}{\displaystyle}
\newcommand{\N}{\mathbb N}
\newcommand{\B}{\mathbf B}
\newcommand{\Z}{\mathbb Z}
\newcommand{\Q}{\mathbb Q}
\newcommand{\R}{\mathbb R}
\def\C{\mathbb C}
\def\U{\mathcal U}
\newcommand{\pow}{\mathcal P}
\newcommand{\inv}{^{-1}}
\newcommand{\st}{:}
\renewcommand{\iff}{\leftrightarrow}
\newcommand{\Iff}{\Leftrightarrow}
\newcommand{\imp}{\rightarrow}
\newcommand{\Imp}{\Rightarrow}
\newcommand{\isom}{\cong}

\renewcommand{\bar}{\overline}
\newcommand{\card}[1]{\left| #1 \right|}
\newcommand{\twoline}[2]{\begin{pmatrix}#1 \\ #2 \end{pmatrix}}

\newcommand{\vtx}[2]{node[fill,circle,inner sep=0pt, minimum size=4pt,label=#1:#2]{}}
\newcommand{\va}[1]{\vtx{above}{#1}}
\newcommand{\vb}[1]{\vtx{below}{#1}}
\newcommand{\vr}[1]{\vtx{right}{#1}}
\newcommand{\vl}[1]{\vtx{left}{#1}}
\renewcommand{\v}{\vtx{above}{}}

%% Graphics Preamble Entries
\usepackage{tikz, pgfplots}

\usetikzlibrary{positioning,matrix,arrows}

\usetikzlibrary{shapes,decorations,shadows,fadings,patterns}
\usetikzlibrary{decorations.markings}

\usepackage{skak} %for chessboards etc.

\def\circleA{(-.5,0) circle (1)}
\def\circleAlabel{(-1.5,.6) node[above]{$A$}}
\def\circleB{(.5,0) circle (1)}
\def\circleBlabel{(1.5,.6) node[above]{$B$}}
\def\circleC{(0,-1) circle (1)}
\def\circleClabel{(.5,-2) node[right]{$C$}}
\def\twosetbox{(-2,-1.4) rectangle (2,1.4)}
\def\threesetbox{(-2.5,-2.4) rectangle (2.5,1.4)}
\newcommand{\hexbox}[3]{
  \def\x{-cos{30}*\r*#1+cos{30}*#2*\r*2}
  \def\y{-\r*#1-sin{30}*\r*#1}
  \draw (\x,\y) +(90:\r) -- +(30:\r) -- +(-30:\r) -- +(-90:\r) -- +(-150:\r) -- +(150:\r) -- cycle;
  \draw (\x,\y) node{#3};
}

\tikzset{->-/.style={decoration={
  markings,
  mark=at position .5 with {\arrow{>}}},postaction={decorate}}}

  \newcommand{\onedot}{
    +(.5,.5) \v
  }
  \newcommand{\twodots}{
    +(.25,.25) \v +(.75,.75) \v
  }
  \newcommand{\threedots}{
  +(.25,.25) \v +(.5, .5) \v +(.75,.75) \v
  }
  \newcommand{\fourdots}{
    +(.25,.25) \v +(.25,.75) \v +(.75,.25) \v +(.75,.75) \v
  }
  \newcommand{\fivedots}{
    +(.5,.5) \v +(.25,.25) \v +(.25,.75) \v +(.75,.25) \v +(.75,.75) \v
  }
  \newcommand{\sixdots}{
    +(.25,.5) \v +(.75,.5) \v +(.25,.25) \v +(.25,.75) \v +(.75,.25) \v +(.75,.75) \v
  }
  \newcommand{\dominoborder}{
    \draw[thick, rounded corners] (0,0) rectangle (1,2);
    \draw[thin] (0,1) -- (1,1);
  }


%%%% End of PreTeXt generated preamble %%%%% 

\title{Combinations and Permutations}
\subtitle{(Section 1.3)}
\author{}
\date[]{}

\begin{document}
\begin{frame}
\maketitle 
\end{frame}
 
\begin{frame}
\frametitle{Overview}
\tableofcontents 
\end{frame}
 
\begin{frame}
\frametitle{Investigate!}
 You have a bunch of chips which come in five different colors: red, blue, green, purple and yellow.\begin{enumerate}
\item{} How many different two-chip stacks can you make if the bottom chip must be red or blue? Explain your answer using both the additive and multiplicative principles.


\item{} How many different three-chip stacks can you make if the bottom chip must be red or blue and the top chip must be green, purple or yellow? How does this problem relate to the previous one?


\item{} How many different three-chip stacks are there in which no color is repeated? What about four-chip stacks?


\item{} Suppose you wanted to take three different colored chips and put them in your pocket. How many different choices do you have? What if you wanted four different colored chips? How do these problems relate to the previous one?

\end{enumerate}

\end{frame}
 
\begin{frame}
\frametitle{}
A \terminology{permutation} is a (possible) rearrangement of objects. For example, there are 6 permutations of the letters \emph{a, b, c}:%
\begin{equation*}
abc, ~~ acb, ~~ bac, ~~bca, ~~ cab, ~~ cba\text{.}
\end{equation*}

 
\pause \vfill 

We know that we have them all: there are 3 choices for which letter we put first, then 2 choices for which letter comes next, which leaves only 1 choice for the last letter. The multiplicative principle says we multiply \(3\cdot 2 \cdot 1\).
\end{frame}
 
\begin{frame}
\frametitle{}
\begin{example}[1.3.1]How many permutations are there of the letters \emph{a, b, c, d, e, f}?
\end{example}
\end{frame}
 
\begin{frame}
\frametitle{Permutations of \(n\) elements}
 There are \(n! = n\cdot (n-1)\cdot (n-2)\cdot \cdots \cdot 2\cdot 1\) permutations of \(n\) (distinct) elements.
\end{frame}
 
\begin{frame}
\frametitle{}
\begin{example}[1.3.2Counting Bijective Functions.]How many functions \(f:\{1,2,\ldots,8\} \to \{1,2,\ldots, 8\}\) are \emph{bijective}?
\end{example}
\end{frame}
 
\begin{frame}
\frametitle{}
\begin{example}[1.3.3]How many 4 letter ``words'' can you make from the letters \emph{a} through \emph{f}, with no repeated letters?
\end{example}
\end{frame}
 
\begin{frame}
\frametitle{}
How many permutations exist of \(k\) objects choosing those objects from a larger collection of \(n\) objects?
 
\pause \vfill 

We write this number \(P(n,k)\) and sometimes call it a \terminology{\(k\)-permutation of \(n\) elements}. To compute \(P(n,k)\) we must apply the multiplicative principle to \(k\) numbers, starting with \(n\) and counting backwards. For example%
\begin{equation*}
P(10, 4) = 10\cdot 9 \cdot 8 \cdot 7\text{.}
\end{equation*}

\end{frame}
 
\begin{frame}
\frametitle{\(k\)-permutations of \(n\) elements}
 \index{\emph{k}-permutation}\index{permutation} \(P(n,k)\) is the number of \terminology{\(k\)-permutations of \(n\) elements}, the number of ways to \emph{arrange} \(k\) objects chosen from \(n\) distinct objects.%
\begin{equation*}
P(n,k) = \frac{n!}{(n-k)!} = n(n-1)(n-2)\cdots (n-(k-1))\text{.}
\end{equation*}

\end{frame}
 
\begin{frame}
\frametitle{}
\begin{example}[1.3.4Counting injective functions.]How many functions \(f:\{1,2,3\} \to \{1,2,3,4,5,6,7,8\}\) are \emph{injective}?
\end{example}
\end{frame}
 
\begin{frame}
\frametitle{}
Another way to find the number of \(k\)-permutations of \(n\) elements:\begin{itemize}
\item{} First \emph{choose} which \(k\) elements will be in the permutation.


\item{} Then arrange them (permute them).

\end{itemize}
Using the multiplicative principle, we get another formula for \(P(n,k)\):%
\begin{equation*}
P(n,k) = {n \choose k}\cdot k!\text{.}
\end{equation*}

 
\pause \vfill 

Divide by \(k!\) to get a formula for \(\binom{n}{k}\).
\end{frame}
 
\begin{frame}
\frametitle{Closed formula for \({n \choose k}\)}
 %
\begin{equation*}
{n \choose k} = \frac{n!}{(n-k)!k!} = \frac{n(n-1)(n-2)\cdots(n-(k-1))}{k(k-1)(k-2)\cdots 1}\text{.}
\end{equation*}

\end{frame}
 
\begin{frame}
\frametitle{}
\begin{example}[1.3.5]You decide to have a dinner party. Even though you are incredibly popular and have 14 different friends, you only have enough chairs to invite 6 of them.
\begin{enumerate}
\item{} How many choices do you have for which 6 friends to invite?


\item{} What if you need to decide not only which friends to invite but also where to seat them along your long table? How many choices do you have then?

\end{enumerate}

\end{example}
\end{frame}
 
\end{document}
