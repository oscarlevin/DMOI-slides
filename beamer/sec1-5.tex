\documentclass[11pt, compress]{beamer}
\usepackage{amsmath}
\usetheme{Boadilla}
\usefonttheme[onlymath]{serif}
%get rid of navigation:
\setbeamertemplate{navigation symbols}{}


 %%%% Start PreTeXt generated preamble: %%%%% 

%% Some aspects of the preamble are conditional,
%% the LaTeX engine is one such determinant
\usepackage{ifthen}
\newcommand{\tabularfont}{}
\usepackage[xparse, raster]{tcolorbox}
\tcbset{colback=white, colframe=white}
\NewTColorBox{image}{mmm}{boxrule=0.25pt, colframe=gray, left skip=#1\linewidth,width=#2\linewidth}
\RenewTColorBox{definition}{m}{colback=teal!30!white, colbacktitle=teal!30!white, coltitle=black, colframe=gray, boxrule=0.5pt, sharp corners=downhill, titlerule = 0.25pt, title={#1}}
\RenewTColorBox{theorem}{m}{colback=pink!30!white, colbacktitle=pink!30!white, coltitle=black, colframe=gray, boxrule=0.5pt, sharp corners=downhill, titlerule = 0.25pt, title={#1}}
\RenewTColorBox{proof}{}{boxrule=0.25pt, colframe=gray, colback=white, before upper={Proof:}, after upper={\qed}}
\newcommand{\lt}{<}
\newcommand{\gt}{>}
\newcommand{\amp}{&}

%% Begin: Semantic Macros
%% To preserve meaning in a LaTeX file
%%
%% \mono macro for content of "c", "cd", "tag", etc elements
%% Also used automatically in other constructions
%% Simply an alias for \texttt
%% Always defined, even if there is no need, or if a specific tt font is not loaded
\newcommand{\mono}[1]{\texttt{#1}}
%%
%% Following semantic macros are only defined here if their
%% use is required only in this specific document
%%
%% Used for inline definitions of terms
\newcommand{\terminology}[1]{\textbf{#1}}
%% End: Semantic Macros

\renewcommand{\d}{\displaystyle}
\newcommand{\N}{\mathbb N}
\newcommand{\B}{\mathbf B}
\newcommand{\Z}{\mathbb Z}
\newcommand{\Q}{\mathbb Q}
\newcommand{\R}{\mathbb R}
\def\C{\mathbb C}
\def\U{\mathcal U}
\newcommand{\pow}{\mathcal P}
\newcommand{\inv}{^{-1}}
\newcommand{\st}{:}
\renewcommand{\iff}{\leftrightarrow}
\newcommand{\Iff}{\Leftrightarrow}
\newcommand{\imp}{\rightarrow}
\newcommand{\Imp}{\Rightarrow}
\newcommand{\isom}{\cong}

\renewcommand{\bar}{\overline}
\newcommand{\card}[1]{\left| #1 \right|}
\newcommand{\twoline}[2]{\begin{pmatrix}#1 \\ #2 \end{pmatrix}}

\newcommand{\vtx}[2]{node[fill,circle,inner sep=0pt, minimum size=4pt,label=#1:#2]{}}
\newcommand{\va}[1]{\vtx{above}{#1}}
\newcommand{\vb}[1]{\vtx{below}{#1}}
\newcommand{\vr}[1]{\vtx{right}{#1}}
\newcommand{\vl}[1]{\vtx{left}{#1}}
\renewcommand{\v}{\vtx{above}{}}

%% Graphics Preamble Entries
\usepackage{tikz, pgfplots}

\usetikzlibrary{positioning,matrix,arrows}

\usetikzlibrary{shapes,decorations,shadows,fadings,patterns}
\usetikzlibrary{decorations.markings}

\usepackage{skak} %for chessboards etc.

\def\circleA{(-.5,0) circle (1)}
\def\circleAlabel{(-1.5,.6) node[above]{$A$}}
\def\circleB{(.5,0) circle (1)}
\def\circleBlabel{(1.5,.6) node[above]{$B$}}
\def\circleC{(0,-1) circle (1)}
\def\circleClabel{(.5,-2) node[right]{$C$}}
\def\twosetbox{(-2,-1.4) rectangle (2,1.4)}
\def\threesetbox{(-2.5,-2.4) rectangle (2.5,1.4)}
\newcommand{\hexbox}[3]{
  \def\x{-cos{30}*\r*#1+cos{30}*#2*\r*2}
  \def\y{-\r*#1-sin{30}*\r*#1}
  \draw (\x,\y) +(90:\r) -- +(30:\r) -- +(-30:\r) -- +(-90:\r) -- +(-150:\r) -- +(150:\r) -- cycle;
  \draw (\x,\y) node{#3};
}

\tikzset{->-/.style={decoration={
  markings,
  mark=at position .5 with {\arrow{>}}},postaction={decorate}}}

  \newcommand{\onedot}{
    +(.5,.5) \v
  }
  \newcommand{\twodots}{
    +(.25,.25) \v +(.75,.75) \v
  }
  \newcommand{\threedots}{
  +(.25,.25) \v +(.5, .5) \v +(.75,.75) \v
  }
  \newcommand{\fourdots}{
    +(.25,.25) \v +(.25,.75) \v +(.75,.25) \v +(.75,.75) \v
  }
  \newcommand{\fivedots}{
    +(.5,.5) \v +(.25,.25) \v +(.25,.75) \v +(.75,.25) \v +(.75,.75) \v
  }
  \newcommand{\sixdots}{
    +(.25,.5) \v +(.75,.5) \v +(.25,.25) \v +(.25,.75) \v +(.75,.25) \v +(.75,.75) \v
  }
  \newcommand{\dominoborder}{
    \draw[thick, rounded corners] (0,0) rectangle (1,2);
    \draw[thin] (0,1) -- (1,1);
  }


%%%% End of PreTeXt generated preamble %%%%% 

\title{Stars and Bars}
\subtitle{(Section 1.5)}
\author{}
\date[]{}

\begin{document}
\begin{frame}
\maketitle 
\end{frame}
 
\begin{frame}
\frametitle{Overview}
\tableofcontents 
\end{frame}
 
\begin{frame}
\frametitle{Investigate!}
 Suppose you have some number of identical Rubik's cubes to distribute to your friends. Imagine you start with a single row of the cubes.\begin{enumerate}
\item{} Find the number of different ways you can distribute the cubes provided:
\begin{enumerate}
\item{} You have 3 cubes to give to 2 people.


\item{} You have 4 cubes to give to 2 people.


\item{} You have 5 cubes to give to 2 people.


\item{} You have 3 cubes to give to 3 people.


\item{} You have 4 cubes to give to 3 people.


\item{} You have 5 cubes to give to 3 people.

\end{enumerate}



\item{} Make a conjecture about how many different ways you could distribute 7 cubes to 4 people. Explain.


\item{} What if each person were required to get \emph{at least one} cube? How would your answers change?

\end{enumerate}

\end{frame}
 
\begin{frame}
\frametitle{Cookies and Kids}
 You have 7 cookies to give to 4 kids. How many ways can you do this?
 
\pause \vfill 

Take a moment to think about how you might solve this problem. You may assume that it is acceptable to give a kid no cookies. Also, the cookies are all identical and the order in which you give out the cookies does not matter.
 
\pause \vfill 

How can we represent outcomes of this counting process?
\end{frame}
 
\begin{frame}
\frametitle{Outcome Representations}
 One outcome: Three cookies to kid A, one cookie to each of kids B and C, and two cookies to kid D.
 
\pause \vfill 

We could represent this outcome as:
\pause 

\begin{itemize}[<+->]
\item{} 3112


\item{} ABAADCD


\item{} AAABCDD


\item{} \textasteriskcentered{}\textasteriskcentered{}\textasteriskcentered{}\textbar{}\textasteriskcentered{}\textbar{}\textasteriskcentered{}\textbar{}\textasteriskcentered{}\textasteriskcentered{}

\end{itemize}

\end{frame}
 
\begin{frame}
\frametitle{Stars and Bars}
 What outcome do each of the following \terminology{stars and bars diagrams} represent?%
\begin{equation*}
|***||****
\end{equation*}
%
\begin{equation*}
******|*||  
\end{equation*}
%
\begin{equation*}
*|*|*|*|*
\end{equation*}

 
\pause \vfill 

Which stars and bars diagrams represent outcomes?  Each exactly once?  How many are there?
 
\pause \vfill 

There are \(\binom{10}{3}\) ways to distribute 7 cookies to 4 kids.
\end{frame}
 
\begin{frame}
\frametitle{}
\begin{example}[1.5.1]Your favorite mathematical ice-cream parlor offers 10 flavors. How many milkshakes could you create using exactly 6, not necessarily distinct scoops? The order you add the flavors does not matter (they will be blended up anyway) but you are allowed repeats. So one possible shake is triple chocolate, double cherry, and mint chocolate chip.
\end{example}
\end{frame}
 
\begin{frame}
\frametitle{}
\begin{example}[1.5.2]How many 7 digit phone numbers are there in which the digits are non-increasing? That is, every digit is less than or equal to the previous one.
\end{example}
\end{frame}
 
\begin{frame}
\frametitle{}
\begin{example}[1.5.3]How many integer solutions are there to the equation%
\begin{equation*}
x_1 + x_2 + x_3 + x_4 + x_5 = 13\text{.}
\end{equation*}

(An \terminology{integer solution} to an equation is a solution in which the unknown must have an integer value.)
\begin{enumerate}
\item{} where \(x_i \ge 0\) for each \(x_i\)?


\item{} where \(x_i > 0\) for each \(x_i\)?


\item{} where \(x_i \ge 2\) for each \(x_i\)?

\end{enumerate}

\end{example}
\end{frame}
 
\begin{frame}
\frametitle{Counting with Functions}
 How is distributing 7 cookies to 4 kids like a function?
 
\pause \vfill 

The number of functions \(f: \{1,2,3,4,5,6,7\} \to \{a,b,c,d\}\) is \(4^7 = 16384\).
 
\pause \vfill 

The number of ways to assign 7 cookies to 4 kids is \({10 \choose 3} = 120\). What is going on here?
 
\pause \vfill 

When we count functions, we consider the following two functions to be different:%
\begin{equation*}
f = \twoline{1 \amp 2 \amp 3 \amp 4\amp 5 \amp 6 \amp 7}{a \amp b \amp c \amp c \amp c \amp c \amp c} \qquad g = \twoline{1 \amp 2 \amp 3 \amp 4\amp 5 \amp 6 \amp 7}{b \amp a \amp c \amp c \amp c \amp c \amp c}\text{.}
\end{equation*}
But these two functions would correspond to the \emph{same} cookie distribution.
\end{frame}
 
\end{document}
